\documentclass[12pt]{amsart}
\usepackage{ytableau}
\newtheorem{theorem}{Theorem}
\theoremstyle{definition}
\newtheorem{definition}[theorem]{Definition}
\newtheorem{example}[theorem]{Example}
\newcommand{\rowins}{\mathrm{ROWINS}}
\newcommand{\ins}{\mathrm{INSERT}}
\newcommand{\Tab}{\mathrm{Tab}}
\title{The Timed Plactic Monoid}
\author{Amritanshu Prasad}
\begin{document}
\maketitle
\begin{abstract}
  The plactic monoid is a simple algebraic structure that lies at the crossroads of the theory of symmetric polynomials, enumerative geometry, representation theory, and combinatorics.
  It was introduced by Lascoux and Sch\"utzenberger, who used it to prove the Littlewood-Richardson rule.
  It is the quotient of the conncatenation-monoid of words in a totally ordered langauge modulo a pair of relations discovered by Knuth.
  Timed words were introduced by Adur and Dill in the context of timed automata, which model the behavior of real-time computer systems.
  Timed words too form a monoid under concatenation.
  This monoid contains the monoid of words as a submonoid.
  In this article, we extend the definition of the plactic monoid to include timed words by presenting timed analogs of Knuth relations.
  We extend some basic results from the theory of plactic monoids, such as Greene's theorem and the RSK correspondence to timed words.
\end{abstract}
\section{Greene's Theorem}
\subsection{Tableaux}
\label{sec:tableaux}
Recall that a partition is a tuple $\lambda=(\lambda_1,\dotsc,\lambda_l)$ of integers such that $\lambda_1\geq \dotsb\geq \lambda_l>0$.
The Young diagram of the partition $\lambda$ is defined as the array of points
\begin{displaymath}
Y(\lambda)=\{(i,j)\mid 1\leq i\leq l,\;1\leq j\leq \lambda_i\}
\end{displaymath}
drawn in matrix notation, so that the point $(i,j)$ lies in the $i$th row and $j$th column of $Y(\lambda)$.
Let $A_n=\{1,\dotsc,n\}$.
\begin{definition}
  A semistandard Young tableau\footnote{It is customary to write the plural of \emph{tableau} as \emph{tableaux}, following French conventions.} in $A_n$ of shape $\lambda$ is an assignment $t:Y(\lambda)\to A_n$ such that the numbers increase weakly from left to right along each row, and from top to bottom along each column.
  The weight of $t$ is the tuple $(m_1,\dotsc, m_n)$, where $m_i$ is the number of times that $i$ occurs in the image of $t$.
\end{definition}
For brevity, a semistandard Young tableaux will be referred to as a tableau in the rest of this article.
\begin{example}
  \label{example:ssyt}
  The following is a tableau of shape $(5,2,1)$ and weight $(2,1,4,1)$ in $A_4$:
  \begin{displaymath}
    t=\ytableaushort{11333,24,3}
  \end{displaymath}
\end{example}
We denote by $\Tab_n$ the set of all tableaux in $A_n$, $\Tab_n(\lambda)$ the set of all tableaux of shape $\lambda$ in $A_n$, and by $\Tab(\lambda,\mu)$ the set of all tableaux of shape $\lambda$ and weight $\mu$.
\subsection{Row Insertion}
\label{sec:row-insertion}A row of length $k$ is defined to be a weakly increasing sequence $u=a_1a_2\dotsb a_k$ in $A_n$.
Let $R(A_n)$ denote the set of all rows in $A_n$.
Each row of a tableau is a row in the sense of this definition.
For each $u=a_1\dotsb a_k\in R(A_n)$ and $a\in A_n$, define:
\begin{displaymath}
  \rowins(u,a) =
  \begin{cases}
    (\emptyset, a_1\dotsb a_k a) & \text{if } a_k\leq a,\\
    (a_j,a_1\dotsb a_{j-1}aa_{j+1}\dotsb a_k) & \text{otherwise, with}\\
    & j=\min\{i\mid a<a_i\}. 
  \end{cases}
\end{displaymath}
Here $\emptyset$ should be thought of as an empty row of length zero.
\begin{example}
  $\rowins(11333,3) = (\emptyset,113333)$, $\rowins(11333,2)=(3,11233)$. 
\end{example}
It is clear from the construction that, for any $u\in R(A_n)$ and $a\in A_n$, if $(a',u')=\rowins(u,a)$, then $u'$ is again a row.
For convenience set $\rowins(u,\emptyset)=(\emptyset,u)$.
\subsection{Tableau Insertion}
\label{sec:tableau-insertion}
Let $t$ be a tableau with rows $u_1,u_2,\dotsc, u_l$.
Then $\ins(t,a)$ is defined as follows: first $a$ is inserted into $u_1$; if $\rowins(u_1,a)=(a_1',u_1')$, then $u_1$ is replaced by $u_1'$.
Then $a_1'$ is inserted into $u_2$; if $\rowins(u_2,a_1')=(a_2',u_3)$, then $u_2$ is replaced by $u_2'$, and so on.
This process continues, generating $a_1',a_2',\dotsc,a_k'$ and $u_1',\dotsc,u_k'$.
The tableau $t'=\ins(t,a)$ has rows $u_1',\dotsc,u_k'$, and a last row (possibly empty) consisting of $a_k'$.
\begin{example}
  \label{example:insertion}
  For $t$ as in Example~\ref{example:insertion}, we have
  \begin{displaymath}
    \ins(t,2) = \ytableaushort{11233,23,34},
  \end{displaymath}
  since $\rowins(11333,2)=(3,11233)$, $\rowins(24,3)=(4,23)$, and $\rowins(3,4)=(\emptyset, 34)$.
\end{example}
\subsection{Insertion Tableau of a Word}
\label{sec:insert-tabl-word}
An arbitrary sequence $a_1\dotsb a_k$ in $A_n$ will be called a word in $A_n$.
The set of all words in $A_n$ is denoted by $A_n^*$.
This set may be regarded as a monoid under concatenation, with identity element as the empty word, denoted by $\emptyset$.
\begin{definition}
\label{definition:insertion-tableau}
The insertion tableau $P(w)$ of a word $w$ is defined recursively as:
\begin{align}
  P(\emptyset)&=\emptyset\\
  P(a_1\dotsb a_k)=\ins(P(a_1\dotsb a_{k-1}), a_k).
\end{align}
\end{definition}
\begin{example}
  \label{example:insertion-tableau}
  Take $w=133324132$.
  Then using Definition~\ref{definition:insertion-tableau}, then sequentially inserting the terms of $w$ into the empty tableau $\emptyset$ gives the sequence of tableaux:
  \begin{displaymath}
    \ytableausetup{smalltableaux}
    \ytableaushort{1},\ytableaushort{13},\ytableaushort{133},\ytableaushort{1333},\ytableaushort{1233,3},\ytableaushort{12334,3},\ytableaushort{11334,2,3},\ytableaushort{11,24,3},\ytableaushort{11233,23,34}
  \end{displaymath}
  the last of which is the insertion tableau $P(w)$.
\end{example}
\subsection{Greene's Theorem}
\label{sec:words}
Given a word $w=a_1a_2\dotsb a_l$, a subword is a word of the form
\begin{displaymath}
  v = a_{i_1}a_{i_2}\dotsb a_{i_k},
\end{displaymath}
for some $1\leq i_1<i_2<\dotsb < i_k$.
We say that the subword $v$ is a row if $a_{i_1}\leq a_{i_2}\leq a_{i_k}$. 
The subword $v$ as above is said to be disjoint from a subword $u=a_{j_1} a_{j_2}\dotsb a_{j_h}$ if the sets $\{i_1,i_2,\dotsc,i_k\}$ and $\{j_1,j_2,\dotsc,j_h\}$ of indices are disjoint.

Given a word $w$, its $k$th Greene invariant \emph{Greene invariant} $a_k(w)$ are defined as the maximal cardinality of a union of $k$ pairwise disjoint row subwords.

Schensted~\cite{schensted} showed that the first Greene invariant $a_1(w)$ of a word is the length of the first row of its insertion tableau $P(w)$.
For instance, the word $w$ from Example~\ref{example:insertion-tableau} has longest increasing row subword of length $5$, and its insertion tableau has first row of length $5$.

\begin{theorem}
  [Greene~\cite{Greene-schen}]
  \label{theorem:Greene}
  For any $w\in A_k$, suppose that the insertion tableau $P(w)$ has $l$ rows of length $\lambda_1,\dotsc,\lambda_l$.
  Then, for each $k=1,\dotsc,l$, $a_l(w)=\lambda_1+\dotsb + \lambda_k$.
\end{theorem}
\subsection{Knuth Relations and the Plactic Monoid}
\label{sec:knuth-equivalence}
The most elegant proof of Greene's theorem (Theorem~\ref{theorem:Greene}) proceeds via the notions of Knuth equivalence and the plactic monoid (see \cite{Lascoux}).

The plactic monoid $(A_n)$ is the quotient of the monoid $A_n^*$ by the submonoid generated by the Knuth relations:
\begin{gather}
  \tag{$K1$}\label{eq:k1}
  xzy \equiv zxy \text{ if } x\leq y < z,
  \\
  \tag{$K2$}\label{eq:k2}
  yxz \equiv yzx \text{ if } x < y \leq z.
\end{gather}
On a more concrete level, it is the set of words $w\in A_n$ modulo equivalence, were two words $v$ and $w$ are deemed to be equivelent if $w$ can be obtained from $v$ by a sequence of moves of the form (\ref{eq:k1}) and (\ref{eq:k2}) involving any three letters of the words obtained at each stage.
For example,
\begin{displaymath}
  11232\equiv_{K1} 11322\equiv_{K1} 13122 \equiv_{K1} 
\end{displaymath}


\bibliographystyle{abbrv}
\bibliography{refs}
\end{document}
