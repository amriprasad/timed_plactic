\documentclass[12pt]{amsart}
\usepackage{color,ytableau}
\newtheorem{theorem}{Theorem}[subsection]
\newtheorem{lemma}[theorem]{Lemma}
\theoremstyle{definition}
\newtheorem{definition}[theorem]{Definition}
\newtheorem{example}[theorem]{Example}
\newcommand{\rowins}{\mathrm{ROWINS}}
\newcommand{\ins}{\mathrm{INSERT}}
\newcommand{\Tab}{\mathrm{Tab}}
\newcommand{\rc}[1]{\mathbf{#1}}
\newcommand{\rd}{\mathrm{read}}
\newcommand{\wt}{\mathrm{wt}}
\newcommand{\shape}{\mathrm{shape}}
\newcommand{\pl}{\mathrm{pl}}
\title{The Timed Plactic Monoid}
\author{Amritanshu Prasad}
\address{The Institute of Mathematical Sciences, Chennai}
\address{Homi Bhabha National Institute, Mumbai}
\begin{document}
\maketitle
\begin{abstract}
  The plactic monoid is a simple algebraic structure that lies at the crossroads of the theory of symmetric polynomials, enumerative geometry, representation theory, and combinatorics.
  It was introduced by Lascoux and Sch\"utzenberger, who used it to prove the Littlewood-Richardson rule.
  It is the quotient of the conncatenation-monoid of words in a totally ordered langauge modulo a pair of relations discovered by Knuth.
  Timed words were introduced by Adur and Dill in the context of timed automata, which model the behavior of real-time computer systems.
  Timed words too form a monoid under concatenation.
  This monoid contains the monoid of words as a submonoid.
  In this article, we extend the definition of the plactic monoid to include timed words by presenting timed analogs of Knuth relations.
  We extend some basic results from the theory of plactic monoids, such as Greene's theorem and the RSK correspondence to timed words.
\end{abstract}
\section{Tableaux, Insertion, and Greene's Theorem}
\label{sec:tabl-insert-green}
\subsection{Tableaux}
\label{sec:tableaux}
Recall that a partition is a tuple $\lambda=(\lambda_1,\dotsc,\lambda_l)$ of integers such that $\lambda_1\geq \dotsb\geq \lambda_l>0$.
The Young diagram of the partition $\lambda$ is defined as the array of points
\begin{displaymath}
Y(\lambda)=\{(i,j)\mid 1\leq i\leq l,\;1\leq j\leq \lambda_i\}
\end{displaymath}
drawn in matrix notation, so that the point $(i,j)$ lies in the $i$th row and $j$th column of $Y(\lambda)$.
Let $A_n=\{1,\dotsc,n\}$.
\begin{definition}
  A semistandard Young tableau\footnote{It is customary to write the plural of \emph{tableau} as \emph{tableaux}, following French conventions.} in $A_n$ of shape $\lambda$ is an assignment $t:Y(\lambda)\to A_n$ such that the numbers increase weakly from left to right along each row, and increase strictly from top to bottom along each column.
  The weight of $t$ is the tuple $(m_1,\dotsc, m_n)$, where $m_i$ is the number of times that $i$ occurs in the image of $t$.
\end{definition}
For brevity, a semistandard Young tableaux will be referred to as a tableau in the rest of this article.
\begin{example}
  \label{example:ssyt}
  The following is a tableau of shape $(5,2,1)$ and weight $(2,1,4,1)$ in $A_4$:
  \begin{displaymath}
    t=\ytableaushort{11333,24,3}
  \end{displaymath}
\end{example}
We denote by $\Tab_n$ the set of all tableaux in $A_n$, $\Tab_n(\lambda)$ the set of all tableaux of shape $\lambda$ in $A_n$, and by $\Tab(\lambda,\mu)$ the set of all tableaux of shape $\lambda$ and weight $\mu$.
\subsection{Row Insertion}
\label{sec:row-insertion}A row of length $k$ is defined to be a weakly increasing sequence $u=a_1a_2\dotsb a_k$ in $A_n$.
Let $R(A_n)$ denote the set of all rows in $A_n$.
Each row of a tableau is a row in the sense of this definition.
For each $u=a_1\dotsb a_k\in R(A_n)$ and $a\in A_n$, define:
\begin{displaymath}
  \rowins(u,a) =
  \begin{cases}
    (\emptyset, a_1\dotsb a_k a) & \text{if } a_k\leq a,\\
    (a_j,a_1\dotsb a_{j-1}aa_{j+1}\dotsb a_k) & \text{otherwise, with}\\
    & j=\min\{i\mid a<a_i\}.
  \end{cases}
\end{displaymath}
Here $\emptyset$ should be thought of as an empty row of length zero.
\begin{example}
  $\rowins(11333,3) = (\emptyset,113333)$, $\rowins(11333,2)=(3,11233)$.
\end{example}
It is clear from the construction that, for any $u\in R(A_n)$ and $a\in A_n$, if $(a',u')=\rowins(u,a)$, then $u'$ is again a row.
For convenience set $\rowins(u,\emptyset)=(\emptyset,u)$.
\subsection{Tableau Insertion}
\label{sec:tableau-insertion}
Let $t$ be a tableau with rows $u_1,u_2,\dotsc, u_l$.
Then $\ins(t,a)$, the insertion of $a$ into $t$, is defined as follows: first $a$ is inserted into $u_1$; if $\rowins(u_1,a)=(a_1',u_1')$, then $u_1$ is replaced by $u_1'$.
Then $a_1'$ is inserted into $u_2$; if $\rowins(u_2,a_1')=(a_2',u_3)$, then $u_2$ is replaced by $u_2'$, and so on.
This process continues, generating $a_1',a_2',\dotsc,a_k'$ and $u_1',\dotsc,u_k'$.
The tableau $t'=\ins(t,a)$ has rows $u_1',\dotsc,u_k'$, and a last row (possibly empty) consisting of $a_k'$.
It turns out that $\ins(t,a)$ is a tableau \cite{knuth}.
\begin{example}
  \label{example:insertion}
  For $t$ as in Example~\ref{example:insertion}, we have
  \begin{displaymath}
    \ins(t,2) = \ytableaushort{11233,23,34},
  \end{displaymath}
  since $\rowins(11333,2)=(3,11233)$, $\rowins(24,3)=(4,23)$, and $\rowins(3,4)=(\emptyset, 34)$.
\end{example}
\subsection{Insertion Tableau of a Word}
\label{sec:insert-tabl-word}
An arbitrary sequence $a_1\dotsb a_k$ in $A_n$ will be called a word in $A_n$.
The set of all words in $A_n$ is denoted by $A_n^*$.
This set may be regarded as a monoid under concatenation, with identity element as the empty word, denoted by $\emptyset$.
\begin{definition}
\label{definition:insertion-tableau}
The insertion tableau $P(w)$ of a word $w$ is defined recursively as:
\begin{align}
  P(\emptyset)&=\emptyset\\
  P(a_1\dotsb a_k)=\ins(P(a_1\dotsb a_{k-1}), a_k).
\end{align}
\end{definition}
\begin{example}
  \label{example:insertion-tableau}
  Take $w=133324132$.
  Then using Definition~\ref{definition:insertion-tableau}, then sequentially inserting the terms of $w$ into the empty tableau $\emptyset$ gives the sequence of tableaux:
  \begin{displaymath}
    \ytableausetup{smalltableaux}
    \ytableaushort{1},\ytableaushort{13},\ytableaushort{133},\ytableaushort{1333},\ytableaushort{1233,3},\ytableaushort{12334,3},\ytableaushort{11334,2,3},\ytableaushort{11333,24,3},
  \end{displaymath}
  and finally, the insertion tableau $P(w)=\ytableaushort{11233,23,34}$.
\end{example}
\subsection{Greene's Theorem}
\label{sec:words}
Given a word $w=a_1a_2\dotsb a_l$, a subword is a word of the form
\begin{displaymath}
  v = a_{i_1}a_{i_2}\dotsb a_{i_k},
\end{displaymath}
for some $1\leq i_1<i_2<\dotsb < i_k$.
We say that the subword $v$ is a row if $a_{i_1}\leq a_{i_2}\leq a_{i_k}$.
The subword $v$ as above is said to be disjoint from a subword $u=a_{j_1} a_{j_2}\dotsb a_{j_h}$ if the sets $\{i_1,i_2,\dotsc,i_k\}$ and $\{j_1,j_2,\dotsc,j_h\}$ of indices are disjoint.

Given a word $w$, its $k$th Greene invariant \emph{Greene invariant} $a_k(w)$ are defined as the maximal cardinality of a union of $k$ pairwise disjoint row subwords.

Schensted~\cite{schensted} showed that the first Greene invariant $a_1(w)$ of a word is the length of the first row of its insertion tableau $P(w)$.
For instance, the word $w$ from Example~\ref{example:insertion-tableau} has longest increasing row subword of length $5$, and its insertion tableau has first row of length $5$.
His theorem was generalized by Curtis Greene:
\begin{theorem}
  [Greene~\cite{Greene-schen}]
  \label{theorem:Greene}
  For any $w\in A_k$, suppose that the insertion tableau $P(w)$ has $l$ rows of length $\lambda_1,\dotsc,\lambda_l$.
  Then, for each $k=1,\dotsc,l$, $a_l(w)=\lambda_1+\dotsb + \lambda_k$.
\end{theorem}
\subsection{Knuth Relations and the Plactic Monoid}
\label{sec:knuth-equivalence}
The most elegant proof of Greene's theorem (Theorem~\ref{theorem:Greene}) proceeds via the notions of Knuth equivalence and the plactic monoid (see \cite{Lascoux}).

The plactic monoid $\pl_0(A_n)$ is the quotient of the monoid $A_n^*$ by the submonoid generated by the Knuth relations:
\begin{gather}
  \tag{$K1$}\label{eq:k1}
  xzy \equiv zxy \text{ if } x\leq y < z,
  \\
  \tag{$K2$}\label{eq:k2}
  yxz \equiv yzx \text{ if } x < y \leq z.
\end{gather}
On a more concrete level, it is the set of words $w\in A_n$ modulo Knuth equivalence, were words $v$ and $w$ are said to be Knuth equivelent if $w$ can be obtained from $v$ by a sequence of moves of the form (\ref{eq:k1}) and (\ref{eq:k2}) involving any three letters of the words obtained at each stage.
For example,
\begin{displaymath}
  113\rc{332}\equiv_{K2} 11\rc{332}3\equiv_{K2} 1\rc{132}33 \equiv_{K1} \rc{131}233 \equiv_{K1} 311233.
\end{displaymath}
At each stage, the letters to which the Knuth moves will be applied to obtain the next stage are highlighted.
\begin{definition}
  If $t$ is a tableau with rows $u_1,\dotsc,u_l$, then its reading word is obtained by concatenating its rows, starting from bottom to top: $\rd(t) = u_lu_{l-1}\dotsb u_1$.
\end{definition}
\begin{example}
  The reading word of the tableau $t$ from Example~\ref{example:ssyt} is $32411333$.
\end{example}
A tableau is easily recovered from the reading word.
Line breaks are inserted after each letter that is followed by a strictly smaller one.
For example, $32411333$ is broken as $3/24/1133$, recovering the rows of the tableau $t$.
However, it is easy to construct examples of words which are not reading words of tableau.
Thus $\rd:\Tab_n\to A_n^*$ is an injective function.
Following Lascoux, Leclerc and Thibon \cite{Lascoux} tableaux are identified with their reading words, and a word in $A_n^*$ is called a tableau if it lies in the image of $\rd$.
\begin{theorem}
  \label{theorem:unique-tab}
  Every word $w\in A_n^*$ is Knuth equivalent to the reading word of $P(w)$.
  Moreover, if $t,t'\in \Tab_n$ have $\rd(t)\equiv \rd(t')$, then $t=t'$.
  Consequently, $P(w)$ is the unique tableau in the Knuth-equivalence class of $w$.
\end{theorem}
The plactic proof of Greene's theorem proceeds via Theorem~\ref{theorem:unique-tab}---it is easy so see that Greene invariants are unchanged under Knuth moves, and that if $t\in \Tab_n(\lambda)$, where $\lambda=(\lambda_1,\dotsc,\lambda_l)$, then
\begin{displaymath}
  a_k(\rd(t)) = \lambda_1+\dotsb +\lambda_k\text{ for }k=1,\dotsc,l.
\end{displaymath}
Theorems \ref{theorem:Greene} and~\ref{theorem:unique-tab} are special cases of their timed versions which will be proved in Section~\ref{sec:timed-vers-green}, inspired by the proof in \cite{Lascoux}.
\section{A Timed Version of Greene's Theorem}
\label{sec:timed-vers-green}
\subsection{Timed Tableaux}
\label{sec:timed-tableaux}
\begin{definition}
  [Timed Word]
  \label{definition:timed-word}
  A timed word of length $r$ in the alphabet $A$ is a peicewise-constant right-continuous function $w:[0,r)\to A_n$.
  We write $l(w)=r$.
  In other words, for some finite sequence $0=r_0<r_1<\dotsc<r_k=r$ of transition points, and letters $c_1,\dotsc, c_k$ in $A$, $w(x) = c_i$ if $r_{i-1}\leq x < r_i$.
  Given such a function, we write
  \begin{equation}
    \label{eq:exp_not}
    w = c_1^{t_1} c_2^{t_2}\dotsb c_k^{t_k}, \text{ },
  \end{equation}
  where $t_i = r_i-r_{i-1}$.
  We call this the \emph{exponential string} for $w$.
\end{definition}
The exponential string, as defined above, is not unique; if two successive letters $c_i$ and $c_{i+1}$ are equal, then we can merge them, replacing $c_i^{t_i}c_{i+1}^{t_{i+1}} = c_i^{t_i+t_{i+1}}$.

The above definition is a finite variant of Definition~3.1 of Alur and Dill~\cite{alur-dill}, where $r=\infty$, and there is an infinite increasing sequence of transition points.

Given timed words $w_1$ and $w_2$, their \emph{concatenation} is defined in the most obvious manner---their exponential strings are concatenated (and if necessary, successive equal values merged).
The monoid formed by all timed words in an alphabet $A$, with product defined by concatenation, will be denoted by $A^\dagger$.
We take $A$ to be $A_n=\{1,\dotsc,n\}$.
The submonoid of $A_n^\dagger$ consisting of timed words where the exponents $t_1,t_2,\dotsc,t_k$ in exponential string (\ref{eq:exp_not}) are integers is the free monoid $A_n^*$ from Section~\ref{sec:tabl-insert-green}.
In fact, all definitions and theorems in this section will specialize to those of Section~\ref{sec:tabl-insert-green} when the exponents are integral.

\begin{definition}
  [Timed Subword]
  Given a timed word $w:[0,r)\to A_n$, and $S\subset [0,r)$ a finite disjoint union of intervals of the form $[a, b)\subset [0,r)$, the timed subword of $w$ with respect to $S$ is defined as the timed word:
  \begin{displaymath}
    w_S(t) = w(\inf\{u\in [0,r)\mid \mathrm{meas}([0,u)\cap S) \geq t\}) \text{ for } 0\leq t < \mathrm{meas}(S).
  \end{displaymath}
  Given two words $v$ and $w$, $v$ is said to be a subword of $w$ if there exists $S\subset [0,r)$ as above such that $v=w_S$.
  Subwords $v_1,\dotsc,v_k$ of $w$ are said to be pairwise disjoint if there exist pairwise disjoint subsets $S_1,\dotsc,S_k$ as above such that $v_i=w_{S_i}$ for $i=1,\dotsc,k$.
\end{definition}

A \emph{timed row} is, by definition, a weakly increasing timed word.
Every timed word $w$ has a unique decomposition into rows:
\begin{displaymath}
  w = u_l u_{l-1}\dotsb u_1,
\end{displaymath}
such $u_i$ is a row for each $i=1,\dotsc,l$, and $u_iu_{i-1}$ is not a row for any $i=2,\dotsc,l$.
We shall refer to such a decomposition as the row decomposition of $w$.
Given two rows $u$ and $v$, say that $u$ is dominated by $v$ (denoted $u\lhd v$) if $l(u)\geq l(v)$ and $u(t)<v(t)$ for all $0\leq t<l(v)$.
\begin{definition}[Timed Tableau]\label{definition:timed-tableau}
  A timed tableau in $A_n$ is a timed word $w$ in $A_n$ with row decomposition $w=u_l u_{l-1}\dotsb u_1$ such that $u_1\lhd \dotsb \lhd u_l$.
  The shape of $w$ is the weakly decreasing tuple $(l(u_1),l(u_2),\dotsc,l(u_l))$ of positive real numbers (henceforth called a \emph{real partition}), and the wieght of $w$ is the vector:
  \begin{displaymath}
    \wt(w) = (m_1,\dotsc,m_n),
  \end{displaymath}
where $m_i$ is the Lebesgue measure of the pre-image of $i$ under $t$, i.e., $m_i=\mathrm{meas}(w^{-1}(i))$.
\end{definition}
The above is a direct generalization of the notion of the reading word of a tableau in the sense of Section~\ref{sec:tabl-insert-green}.
\begin{example}
  \label{example:timed-tableau}
  $w=3^{0.8}4^{1.1}1^{1.4}2^{1.6}3^{0.7}$ is a timed tableau in $A_5$ of shape $(3.7,1.9)$ and weight $(1.4, 1.6, 1.5, 1.1,0)$.
\end{example}
\subsection{Timed Insertion}
\label{sec:timed-insertion}
Given a timed word $w$ and $0\leq a < b \leq l(w)$, write $w_{[a, b)}$ for the timed word of length $b-a$ such that
\begin{displaymath}
  w_{[a, b)}(t) = w(a+ t) \text{ for } 0\leq t<b-a.
\end{displaymath}
We call $w_{[a,b)}$ a segment of $w$.
If $a=0$, then $w_{[a,b)}$ is called an initial segment of $w$.
\begin{definition}[Timed row insertion]
  \label{definition:timed-row-insertion}
  Given a timed row $w$, define the insertion $\rowins(w, c^{t_c})$ of $c^{t_c}$ into $w$ as follows: if $w(t)\leq c$ for all $0\leq t < l(u)$, then
  \begin{displaymath}
    \rowins(w, c^{t_c}) = (\emptyset, wc^{t_c}).
  \end{displaymath}
  Otherwise, there exists $0\leq t < l(u)$ such that $w(t)>c$.
  Let
  \begin{displaymath}
    t_0 = \min\{0\leq t< u(l)\mid w(t)> c\}.
  \end{displaymath}
  Define
  \begin{displaymath}
    \rowins(w, c^{t_c}) =
    \begin{cases}
      (w_{[t_0, t_0+t_c)}, w_{[0, t_0)}c^{t_c} w_{[t_0+t_c, l(w))}) & \text{if } l(u) - t_0 > t_c,\\
      (w_{[t_0, l(u))}, w_{[0, t_0)} c^{t_c}) & \text{if } l(u) - t_0 \leq t_c.
    \end{cases}
  \end{displaymath}
  If $u$ is a row of the form $c_1^{t_1}\dotsb c_l^{t_l}$.
  Define $\rowins(w,u)$ by induction on $l$ as follows:
  Having defined $(v',w')=\rowins(w,c_1^{t_1}\dotsb c_{l-1}^{t_{l-1}})$,
  let $(v'',w'')=\rowins(w',c_l^{t_l})$.
  Then define
  \begin{displaymath}
    \rowins(w,u) = (v'v'', w'').
  \end{displaymath}
\end{definition}
\begin{example}
  \label{example:timed-row-ins}
  $\rowins(1^{1.4}2^{1.6}3^{0.7},1^{0.7}2^{0.2})=(2^{0.7}3^{0.2},1^{2.1}2^{1.1}3^{0.5})$.
\end{example}
\begin{definition}
  [Timed Tableau Insertion]
  Let $w$ be a timed tableau with row decomposition $u_l\dotsc u_1$, and let $v$ be a timed row.
  Then $\ins(w, v)$, the insertion of $v$ into $w$ is defined as follows:
  first $v$ is inserted into $u_1$.
  If $\rowins(u_1,v)=(v_1',u_1')$, then $v_1'$ is inserted into $u_2$; if $\rowins(u_2,v_1')=(v_2',u_2')$, then $v_2'$ is inserted in $u_3$, and so on.
  This process continues, generating $v_1',\dotsc,v_l'$ and $u_1',\dotsc,u_l'$.
  $\ins(t,v)$ is defined to be $v_l'u_l'\dotsb u_1'$.
  Note that it is quite possible that $v_l'=\emptyset$.
\end{definition}
\begin{example}
  If $w$ is the timed tableau from Example~\ref{example:timed-tableau}, then
  \begin{displaymath}
    \ins(w,1^{0.7}2^{0.2})=3^{0.7}4^{0.2}2^{0.7}3^{0.3}4^{0.9}1^{2.1}2^{1.1}3^{0.5}.
  \end{displaymath}
\end{example}
\begin{theorem}
  \label{theorem:tableauness-of-insertion}
  For any timed tableau $w$ in $A_n$ and any timed row $v$ in $A_n$, $\ins(w,v)$ is again a timed tableau in $A_n$.
  We have
  \begin{displaymath}
    \wt(\ins(w,v)) = \wt(w) + \wt(v),
  \end{displaymath}
  and $\shape(w)$ interleaves $\shape(\ins(w,v))$.
\end{theorem}
The proof of Theorem~\ref{theorem:tableauness-of-insertion} can be broken down into three preparatory lemmas:
\begin{lemma}
  \label{lemma:dom1.5}
  Suppose $u$ and $v$ are rows, and $(v',u')=\rowins(u,v)$.
  Then $v\lhd v'$.
\end{lemma}
\begin{proof}
  Going through the cases in Definition~\ref{definition:timed-row-insertion} shows that $v'$ is a concatenation of segments of $u$, each of which is displaced by a term in the exponential string of an initial segment of $v$.
  Moreover, the value of the term in $v$ is stricly less than the minimum value of the segment that it displaces.
  This shows that $v'\lhd v$.
\end{proof}
\begin{lemma}
  \label{lemma:dom1}
  Suppose $u$ and $v$ are rows, and $(v',u')=\rowins(u,v)$.
  Then $u'\lhd v$.
\end{lemma}
\begin{proof}
  From the argument in the proof of Lemma~\ref{lemma:dom1.5}, it follows that the row $u'$ contains a subword $v''$ that equals the initial segment of $v$ of length $l(v')$.
  By Lemma~\ref{lemma:dom1.5}, $v''(t)<v(t)$ for all $0\leq t<l(v')$, and since $u'$ is a row, and $v''$ is a subword of $u'$, $v''(t)\leq u'(t)$ for all $0\leq t<l(v')$.
  It follows that $u'\lhd v$.
\end{proof}
\begin{lemma}
  \label{lemma:dom2}
  Suppose $u_1\lhd u_2$ and $v_1\lhd v_2$, $(v_1',u_1')=\rowins(u_1,v_1)$, and $(v_2',u_2')=\rowins(u_2,v_2)$.
  Then $u_1'\lhd u_2'$ and $v_1'\lhd v_2'$.
\end{lemma}
\begin{proof}[Proof of Theorem~\ref{theorem:tableauness-of-insertion}]
  Suppose $w$ has row decomposition $u_l\dotsb u_1$.
  Let $(v_1',u_1')=\rowins(u_1,v)$.
  By Lemma~\ref{lemma:dom1}, $v\lhd v_1'$.
  Now suppose $(v_2',u_2')=\rowins(u_2,v_1')$.
  By Lemma~\ref{lemma:dom2}, $v_1'\lhd v_2'$, and $u_1'\lhd u_2'$.
  Continuing in this manner, since $(v_{i+1}',u_{i+1}')=\rowins(u_{i+1},v_i')$, it follows that $u_i'\lhd u_{i+1}'$ for $i=1,\dotsc,l-1$ by induction on $i$.
  Finally, it remains to show that $u_l'\lhd v_l'$, but this follows from Lemma~\ref{lemma:dom1.5}.
\end{proof}
\begin{definition}
  [Insertion Tableau of a Timed Word]
  Let $w$ be a timed word with row decompositon $u_1\dotsb u_l$.
  The insertion tableau of $w$ is defined as:
  \begin{displaymath}
    P(w) = \ins(\dotsb\ins(\ins(u_1, u_2),u_3),\dotsc,u_l).
  \end{displaymath}
\end{definition}
\begin{example}
  If $w=3^{0.8}1^{0.5}4^{1.1}1^{0.9}2^{1.6}3^{0.7}1^{0.7}2^{0.2}$ has four rows in its row decomposition.
  $P(w)$ is calculated via the following steps:
  \begin{displaymath}
    \begin{array}{|l|l|}
      \hline
      w & P(w)\\
      \hline
      3^{0.8} & 3^{0.8}\\
%      3^{0.8}1^{0.5} & 3^{0.5}1^{0.5}3^{0.3}\\
      3^{0.8}1^{0.5}4^{1.1} & 3^{0.5}1^{0.5}3^{0.3}4^{1.1}\\
%      3^{0.8}1^{0.5}4^{1.1}1^{0.9} & 3^{0.8}4^{0.6}1^{1.4}4^{0.5}\\
%      3^{0.8}1^{0.5}4^{1.1}1^{0.9}2^{1.6} & 3^{0.8}4^{1.1}1^{1.4}2^{1.6}\\
      3^{0.8}1^{0.5}4^{1.1}1^{0.9}2^{1.6}3^{0.7} & 3^{0.8}4^{1.1}1^{1.4}2^{1.6}3^{0.7}\\
%      3^{0.8}1^{0.5}4^{1.1}1^{0.9}2^{1.6}3^{0.7}1^{0.7} & 3^{0.7}2^{0.7}3^{0.1}4^{1.1}1^{2.1}2^{0.9}3^{0.7}\\
      3^{0.8}1^{0.5}4^{1.1}1^{0.9}2^{1.6}3^{0.7}1^{0.7}2^{0.2} & 3^{0.7}4^{0.2}2^{0.7}3^{0.3}4^{0.9}1^{2.1}2^{1.1}3^{0.5}\\
      \hline
    \end{array}
  \end{displaymath}
\end{example}
\subsection{Greene's Invariants for Timed Words}
\label{sec:timed-greene-invar}
\begin{definition}[Greene's Invariants for Timed Words]
  \label{definition:timed-Greene-invars}
  Given a word $w\in A_n^\dagger$, its $k$th Greene's invariant $a_k(w)$ is defined to be the maximum possible sum of lengths of a set of $k$ pairwise disjoint subwords of $w$ whose elements are all rows:
  \begin{multline*}
    a_k(w) = \max\{l(u_1)+\dotsb+l(u_k)\mid u_1,\dotsc,u_k \text{ are pairwise disjoint}\\ \text{subwords and each $u_i$ is a row }\}
  \end{multline*}
\end{definition}
\begin{lemma}
  If $w$ is a timed tableau of shape $\lambda=(\lambda_1,\dotsc,\lambda_l)$, then for each $1\leq k\leq l$,
  \begin{displaymath}
    a_k(w) = \lambda_1+\dotsb + \lambda_k.
  \end{displaymath}
\end{lemma}
\begin{proof}
  This proof is very similar to the proof of the corresponding result in \cite{Lascoux}.
  Indeed, $u_1,\dotsc,u_k$ are pairwise disjoint subwords that are rows, so
  \begin{displaymath}
    a_k(w) \geq \lambda_1+\dotsb + \lambda_l.
  \end{displaymath}
  Conversely, any row subword of $w$ will cannot consist of overlapping segments from two different rows $u_i$ and $u_j$ of $w$, because if $i>j$, then $u_i(t)>u_j(t)$, but in the row decomposition of $w$, $u_i$ occurs before $u_j$.
  Therefore, $k$ disjoint subwords can have length at most the sum of lengths of the largest $k$ rows of $w$, which is $\lambda_1+\dotsc+\lambda_k$.
\end{proof}
\subsection{Timed Knuth Equivalence and the Timed Plactic Monoid}
\begin{definition}
  [Timed Knuth Relations]
  The first timed Knuth relation is, given timed rows $x$, $y$, and $z$:
  \begin{equation}
    \tag{$\kappa_1$}
    \label{eq:tk1}
    xzy \equiv zxy,
  \end{equation}
  if $xyz$ is a row, $l(z) = l(y)$, and the last letter of $y$ is strictly less than the first letter of $z$.

  The second timed Knuth relation is, given timed rows $x$, $y$, and $z$:
  \begin{equation}
    \tag{$\kappa_2$}
    \label{eq:tk2}
    yxz \equiv yzx,
  \end{equation}
  if $xyz$ is a row, $l(x)=l(y)$, and the last letter of $x$ is strictly less than the first letter of $y$.

  The timed plactic monoid $\pl(A_n)$ is the quotient $A^\dagger/\equiv$, where $\equiv$ is the congruence generated by the timed Knuth relations (\ref{eq:tk1}) and (\ref{eq:tk2}).
\end{definition}
For readers not familar with rewriting rules or quotient monoids, it is worth devoting a paragraph to unraveling the last line of the above definition:
two elements of $A^\dagger$ are said to differ by a Knuth move if they are of the form $uv_1w$ and $uv_2w$, where $v_1$ and $v_2$ are terms on opposite sides of one of the timed Knuth relations (\ref{eq:tk1}) and (\ref{eq:tk2}).
Knuth equivalence $\equiv$ is the equivalence relation generated by Knuth moves.
Since this equivalence is stable under left and right multiplication in $A^\dagger$, the concatenation product on $A^\dagger$ descends to a product on the set $\pl(A)$ of Knuth equivalence classes, giving it the structure of a monoid.
\begin{lemma}
  \label{lemma:Knuth-Greene}
  If two timed words are Knuth equivalent, then they have the same Greene invariants.
\end{lemma}
\begin{proof}
  It suffices to prove that if two words differ by a Knuth move they have the same Greene invariants.
  Suppose, for example, that $xyz$ is a timed row with $l(z)=l(y)$, and the last letter of $y$ is strictly less than the first letter of $z$.
  For any timed words $w$ and $u$, we wish to show that Greene's invariants coincide for $wxzyu$ and $wzxyu$.
  Now suppose that $v_1,\dotsc,v_k$ are pairwise disjoint row subwords of $wxzyu$.
  Then since the last letter of $y$ is strictly less than the first letter of $z$, each $v_i$ can either include segments from 
\end{proof}
\label{sec:timed-knuth-equiv}
\bibliographystyle{abbrv}
\bibliography{refs}
\end{document}
